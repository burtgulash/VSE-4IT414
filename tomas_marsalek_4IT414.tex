\documentclass[10pt]{article}
\usepackage[utf8]{inputenc}
\usepackage[czech]{babel}
\usepackage{a4wide}
\usepackage[numbers]{natbib}
\usepackage[pdfborder={0 0 0}]{hyperref}
\usepackage[bottom=2.5cm]{geometry}



\begin{document}
\pagenumbering{gobble}
\thispagestyle{empty}
\begin{center}
    \Large
    \textbf{Jaká vidíte omezení použití principů TOC a CC v řízení projektů IS/ICT?}

    \vspace{.1cm}
    4IT414 Řízení projektů IS/ICT

    ZS 2014-2015

    \vspace{.6cm}
    \large
    \textbf{Tomáš Maršálek}

    \today

    \vspace{1cm}
\end{center}

TOC (Theory of Constraints) je idea původně zamýšlená pro výrobní procesy, o
kterých se dá tvrdit, že jsou jasně definované, kdežto vývoj softwaru je spíše
výroba nehmatatelných ideí a připomíná komplexní dynamický systém.  Definovat
TOC pro vývoj softwaru je složité, protože je obtížné definovat základní
veličiny k tomu potřebné.  TOC vyžaduje definovat míry, podle kterých sledujeme
přinesenou hodnotu. Ve výrobě lehce nadefinujeme produkt jako hotový výrobek.
Ve vývoji SW lze použít pouze vágní jednotky jako např. jeden use case, story
point, nová vlastnost, položka z backlogu nebo podobné.

Pro tyto jednotky je velmi obtížné přesně odhadnout jejich náročnost a užitek.
Odhad náročnosti je složitý, protože se jedná o implementaci nového nápadu a ne
pouze dalšího produktu v rutinní výrobě. Užitek této nové vlastnosti je pouze
předpoklad užitku vlastníka. Skutečný užitek se projeví až v produkčním
nasazení vlastnosti. 

V analogii k výrobnímu procesu, kde se podle TOC snažíme minimalizovat zásoby
na skladě, ve vývoji SW ani nevíme, co považovat za jednotku na skladě. Velké
množství úkolů ve frontě v Kanbanu nebo v backlogu ve Scrumu nevidíme jako nic
špatného - constraint. Pokud se ve Scrumu úkol nestihl do konce sprintu, je
jednoduše zrušen nebo přesunut do následujícího sprintu.

Pokud nepoužijeme TOC přímo na vývoj SW, ale použijeme aplikaci TOC na řízení
projektů - metodu kritického řetězce, narazíme na několik předpokladů. Metoda
kritického řetězce vyžaduje zdroje s možností rychlého přeřazení mezi úkoly
nebo řetězci. Přepnutí mezi úkoly je ale velmi časově nákladné ve vývoji SW,
protože programátor musí najednou v hlavě držet velké množství informací.
Přepnutí kontextu navíc způsobí, že programátor vypustí ze své krátkodobé
paměti úkol, který opustil, a při navrácení musí na spoustu věcí přijít znovu.

Flexibilita v přehazování programátorů mezi úkoly a minimalizace nečinnosti
mohou způsobit nejen zbytečnou ztrátu času kvůli přepínání kontextu, ale
vystavují projekt dlouhodobému riziku jevům \uv{software bloat} (kypění kódu) a
\uv{feature creep}. V softwaru totiž neplatí, že čím více vlastností, tím je
software kvalitnější. Dlouhodobě mohou některé vlastnosti, které nejsou ani
důležité, způsobit přílišnou provázanost softwaru, mohou přidat závislosti na
softwarových knihovnách a další nežádoucí jevy. 
% TODO kanban queues, scrum backlogs





\bibliographystyle{csplainnat}
\bibliography{ref}

\end{document}
